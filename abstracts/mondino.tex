\mypage
\mydate{Wednesday, 14 June 2023}
\mytime{11:30}
\myauthor{MONDINO, Andrea}
\myaffiliation{University of Oxford}
\mytitle{Timelike Ricci bounds and Einstein's theory of gravity in a non smooth setting: an optimal transport approach}
\begin{myabstract}
Optimal transport tools have been extremely powerful to study Ricci curvature, in particular Ricci lower bounds in the non-smooth setting of metric measure spaces (which can be been as a non-smooth extension of Riemannian manifolds). Since the geometric framework of general relativity is the one of Lorentzian manifolds (or space-times), and the Ricci curvature plays a prominent role in Einstein’s theory of gravity, a natural question is whether optimal transport tools can be useful also in this setting. The goal of the talk is to introduce the topic and to report on recent progress. More precisely: After recalling the general setting of Lorentzian synthetic spaces (including important examples fitting the framework), I will discuss some basics of optimal transport theory thereof in order to define ``timelike Ricci curvature and dimension bounds'' for a possibly non-smooth Lorentzian space.\\\relax
Some cases of such bounds have remarkable physical interpretations (like the attractive nature of gravity) and can be used to give a characterisation of the Einstein's equations for a non-smooth space.\\\relax
Based partly on joint work with S. Suhr and partly on joint work with F. Cavalletti.
\end{myabstract}

