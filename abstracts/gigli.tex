\mypage
\mydate{Thursday, 15 June 2023}
\mytime{11:30}
\myauthor{GIGLI, Nicola}
\myaffiliation{SISSA Trieste}
\mytitle{Hyperbolic nonsmooth calculus}
\begin{myabstract}
In the last 25 years, tremendous progresses have been made in the field of non-smooth analysis: after the pioneering works of the Finnish school and Cheeger’s seminal contribution, interest on the topic has been revamped by Lott-Sturm-Villani’s papers on weak lower Ricci curvature bounds.

More recently, there has been a surging interest in non-smooth ``hyperbolic'' geometry, i.e. in spaces whose smooth counterpart are Lorentzian manifolds rather than ``elliptic'' Riemannian ones. Motivations come both from geometry and physics and concern in particular, after works of Cavalletti, McCann, Mondino, Suhr, genuinely non-smooth theories of gravity. These new geometries require new calculus tools: in this talk I will present some partial, but promising, results.

Based on joint works with Beran, Braun, Calisti, McCann, Ohanyan, Rott, Saemann.
\end{myabstract}

