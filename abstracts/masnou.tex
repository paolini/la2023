\mypage
\mydate{Wednesday, 14 June 2023}
\mytime{16:30}
\myauthor{MASNOU, Simon}
\myaffiliation{UCBL Lyon}
\mytitle{Numerical approximation of mean curvature flows using neural networks}
\begin{myabstract}
The talk will focus on new neural network-based numerical methods for approximating the mean curvature flow of general interfaces, both oriented and non-oriented. To learn the correct evolution law, the networks are trained on implicit representations of exact interface evolutions. They have a very simple structure inspired by some splitting schemes used for the discretization of the Allen-Cahn equation. But while the latter schemes only allow to approximate the mean curvature flow of oriented domain boundaries, our networks can also adapt to the non-orientable case. And although trained only on regular flows, they correctly handle different types of singularities. Moreover, they can be easily coupled to various constraints. The talk will show several applications that illustrate the interest of the approach: mean curvature flow with volume constraint, multiphase mean curvature flow, approximation of Steiner trees, approximation of minimal surfaces. The extension to the anisotropic mean curvature flow and the identification of anisotropies will also be discussed. This is a joint work with Elie Bretin, Roland Denis and Garry Terii.
\end{myabstract}
