\mypage
\mydate{Monday, 12 June 2023}
\mytime{15:00}
\myauthor{BRAIDES, Andrea}
\myaffiliation{SISSA, Trieste}
\mytitle{Beyond the classical Cauchy-Born rule}
\begin{myabstract}
Physically motivated variational problems involving non-convex energies are often formulated in a discrete setting and contain boundary conditions.  The long-range interactions in such problems, combined with constraints imposed by lattice discreteness, can give rise to the phenomenon of geometric frustration even in a one-dimensional setting. While non-convexity entails the formation of microstructures, incompatibility between interactions operating at different scales can produce nontrivial mixing effects which are exacerbated in the case of incommensuration between the optimal microstructures and the scale of the underlying lattice. While in general one cannot expect that ground states in such problems possess global properties, such as periodicity, in some cases the appropriately defined ``global'' solutions exist, and are sufficient to describe the corresponding continuum (homogenized) limits. We interpret those cases as complying with a Generalized Cauchy-Born (GCB) rule, and present a new class of problems with geometrical frustration which comply with GCB rule in one range of (loading) parameters while being strictly outside this class in a complimentary range. A general approach to problems with such ``mixed behavior'' is developed. Work in collaboration with A.Causin, M.Solci and L.Truskinovsky.
\end{myabstract}

