\mypage
\mydate{Friday, 16 June 2023}
\mytime{10:30}
\myauthor{CANNARSA, Piermarco}
\myaffiliation{Università degli Studi di Roma Tor Vergata}
\mytitle{Singularities of solutions 
to Hamilton-Jacobi
equations: from PDE’s to topology,
passing through geometric measure theory}
\begin{myabstract}
The study of the structural properties of the set of points 
at which a solution $u$ of a first order Hamilton-Jacobi 
equation fails to be differentiable 
-- in short, the singular set of $u$ -- 
has been the subject of a long-term project 
that started in the late sixties with a seminal paper by 
W.~H.~Fleming. 
Research on such a topic picked up again after the 
introduction of viscosity solutions by M.~Crandall and 
P.-L.~Lions in the eighties and is still ongoing. 
All these years have registered enormous progress in the 
comprehension of the size and structure of singularities. 
First, several authors, including Luigi Ambrosio, 
contributed to develop a fine measure theoretical 
analysis of the singular set. Then, the dynamics 
governing propagation of singularities was identified, 
and connections with weak KAM theory by 
A. Fathi were pointed out. This effort also led to 
interesting topological applications. In this talk, 
I will revisit some of the milestones of the theory 
and describe its recent achievements.
\end{myabstract}
