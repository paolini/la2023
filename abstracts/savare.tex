\mypage
\mydate{Monday, 12 June 2023}
\mytime{11:30}
\myauthor{SAVARÉ, Giuseppe}
\myaffiliation{Università Bocconi, Milano}
\mytitle{Evolution of probability measures:\\beyond gradient flows}
\begin{myabstract}
The theory of contraction semigroups in the Wasserstein space of Euclidean probability measures generated by displacement convex functionals is well understood: the variational approach relies on convergence estimates on the Jordan-Kinderlehrer-Otto
Minimizing Movement scheme and on the non-smooth infinitesimal Riemannian structure of the space. In particular, the contraction property of the trajectories is related to a suitable monotonicity property of the Wasserstein subgradient of the functional. Such monotonicity is also strongly influenced by the Wasserstein metric and involves optimal couplings between measures.
As in Hilbert spaces, it is then natural to investigate the much larger class of evolutions driven by probability vector fields satisfying the metric monotonicity condition. However, the situation, is more complicated due to the lack of the minimizing movement scheme.

In this talk we will give a brief account of the theory, in the case when the driven probability vector field is defined in a set containing a sufficiently rich core of discrete measures, e.g. when the field is defined everywhere.\\\relax
When the dimension of the underlying space is at least 2, the metric monotonicity improves to a much stronger monotonicity property: this allows to use a Lagrangian approach and to apply the Hilbertian theory of maximally monotone operators,\\\relax
to obtain a quite complete and detailed characterization of the evolution and its approximation by the (explicit or implicit) Euler method and by discrete particle systems. Such a Lagrangian perspective also applies to gradient flows and shows that in Wasserstein spaces genuinely dispalcement convex functionals (such as the logarithmic entropy) cannot arise as variational limits of displacement convex and continuous functionals, as typically provided by the Moreau-Yosida regularization in flat spaces.
\end{myabstract}

