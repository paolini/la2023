\mypage
\mydate{Tuesday, 13 June 2023}
\mytime{16:00}
\myauthor{ALBERTI, Giovanni}
\myaffiliation{Università di Pisa}
\mytitle{Frobenius theorem for nonsmooth surfaces}
\begin{myabstract}
A classical result in geometry, Frobenius theorem, states that there exist no \protect $k$-dimensional surface which is tangent to a non-involutive distribution of \protect $k$-planes \protect $V$.\\\relax
One may wonder to which extent this statement can be generalized to weaker notions of surfaces, such as rectifiable sets and currents.

Following the work of Z. Balogh and S. Delladio, an interesting class is that of contact sets, namely sets \protect $E$ where a surface \protect $S$ of class \protect $C^1$ is tangent to the distribution \protect $V$. The first relevant question is whether \protect $E$ must be \protect $k$-negligible. I will show that the answer depends on a combination of the regularity of \protect $S$ and of the boundary of \protect $E$: at one end of the spectrum, if \protect $S$ is of class \protect $C^{1,1}$ then no regularity\\\relax
is required on \protect $E$; at the other end, if \protect $S$ is only of class \protect $C^1$ then \protect $E$ must be in a certain fractional Sobolev class. Counterexamples show that these results are sharp through the entire spectrum.

This is part of an ongoing research project with Annalisa Massaccesi (University of Padova) and Andrea Merlo (University of the Basque Countries).
\end{myabstract}
