\mypage
\mydate{Thursday, 15 June 2023}
\mytime{14:30}
\myauthor{TREVISAN, Dario}
\myaffiliation{Università di Pisa}
\mytitle{Concave Random Assignment Problem in One Dimension: Kantorovich Meets Young}
\begin{myabstract}
We investigate the one-dimensional random assignment problem in the\\\relax
concave case, where the assignment cost is determined by a concave power\\\relax
function of the distance between n source and n target points that are\\\relax
i.i.d.\textbackslash{} random variables. Unlike the convex case, the optimal assignment\\\relax
in the concave case can depend on the power parameter and exhibits a\\\relax
multi-scale structure. We settle some conjectures about its typical\\\relax
properties as the number of sample points increases and prove that the\\\relax
limit of the assignment costs exists if the exponent is not equal to\\\relax
1/2. Our proof employs a novel version of the Monge-Kantorovich problem\\\relax
based on Young integration theory, where the difference between two\\\relax
measures is replaced by the weak derivative of a function with finite\\\relax
q-variation, \protect $q>1$. This approach may be of independent interest. Joint\\\relax
work with M. Goldman.
\end{myabstract}
